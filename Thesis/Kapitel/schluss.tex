\section{Schluss}

\subsection{Fazit}
Entwickelt wurde ein VR-System, welches aus einer CAD-Anwendung und einem körperbasierten Interface besteht und mit der aktuellen HTC Vive Technik getestet und evaluiert wurde. Anwendung und Interface wurden nach wissenschaftlichen Richtlinien, Vergleichen und Einbeziehungen von verwandten Designs und ergonomischen Gegebenheiten des Menschen konzipiert und implementiert. Für die Konzeption wurden Grundlagen zu u.a.~\textit{head-mounted displays}und 3D Benutzungsoberflächen sowie der aktuelle Stand der Technik berücksichtigt. Die Anwendung bietet Bewegungsmechanismen für den Benutzer, sowie Funktionen zur Erstellung, Löschung, Manipulation und Selektion von Objekten. Die Manipulation umfasst die Translation, Rotation und Skalierung von Objekten. Ebenfalls kann die Anwendung durch entsprechendes visuelles Feedback blind bedient und auf Wunsch beendet werden. Das Interface und dessen Funktionen befinden sich auf dem virtuellen Arm des Benutzers und können durch die HTC Vive Controller bedient werden. Die Anwendung benutzt das SteamVR- und FinalIK-Plugin für die Verwendung von VR-Hardware, bzw.~für die Kalibrierung des Benutzers mit dem implementierten virtuellen Dummy-Modell von FinalIK. Die Usability des Systems wurde mit dem System Usability Scale von J.~Brooke überprüft und erreichte dabei eine Wertung von 80,23\%, was einer guten bis exzellenten Usability entspricht. Die getesteten Personen wiesen dabei eine geringe bis sehr gute Erfahrung mit VR auf. Eine hundertprozentige Relation zwischen Erfahrung und erreichtem SUS-Score konnte widerlegt werden.

\subsection{Zukunftsaussichten und Pläne}
Das Ergebnis der Evaluation zeigt, dass sich das entwickelte körperbasierte Interface für eine CAD-Anwendung in VR unter der Verwendung von jungen Erwachsenen durchaus anbietet und sogar im Vergleich zu einem durch Gesten gesteuerten System bessere Ergebnisse bezüglich der Usability aufweist. Der erzielte SUS-Score von 80,23\% bietet jedoch auch Potenzial nach oben. Das System müsste eventuell mit einer größeren Gruppe von Personen getestet werden, um eine stichhaltigere Aussage treffen zu können. Generell denkbar ist jedoch eine deutliche Verbesserung der Usability durch Optimierung des Interfaces und dessen Funktionen. Die Kalibrierung des Benutzers müsste ebenfalls überarbeitet werden, um sich besser an den Körper eines Benutzers anzupassen. Eine Spiegelung des Interfaces und damit eine Linkshänder-Unterstützung, könnte den Eindruck der Usability für Linkshänder erheblich verbessern und die Reichweite des Systems erweitern. Auch können zukünftige Optimierungen und natürlichere Manipulationstechniken die Usability verbessern und die Immersion erhöhen. Erweiterungen der Anwendung, wie das Extrudieren von einzelnen Seiten eines Objektes, das Selektieren mehrerer Objekte und Gruppieren mehrerer ausgewählter Objekte sowie Touchpad-Eingaben für Bewegung und Rotation des Benutzer-Modelles könnten die Anwendung positiv erweitern und die Usability erhöhen. Zukünftig denkbar ist auch eine virtuelle Physik zu implementieren, sodass 3D-Objekte wenn gewünscht auch im physikalisch korrekten Raum getestet und visualisiert werden können.
